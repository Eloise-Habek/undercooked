\chapter{Zaključak i budući rad}
		
		\textbf{\textit{dio 2. revizije}}\\
		
		Grupa undercooked u 17 tjedana uspješno je izradila web aplikaciju za online kuharicu. Faze projekta odvile su se u dvije faze.
        
        U prvoj fazi projekta, okupljen je tim od 7 članova. Najprije je odabrana tehnologija koja će se koristiti tokom projketa te je napravljena podjela projektnih zadataka po interesima i sposobnostima članova, gdje je ujedno bio cilj da svi dođu u dodir sa svim fazama razvoja i da količina posla bude ravnomjerno podijeljena. Rana podjela odgovornosti u timu omogučila je članovima da rano krenu učiti o tehnologiji te istraživati odabrane alate i programskih jezika koju će oni koristiti. Uslijedio je rad na dokumentaciji i pokretanje razvoja aplikacije.
        U drugoj fazi projekta uslijedila je realizacija rješenja i izradba preostale dokumentacije. Kako bi se rad članova što više uskladio i komunikacija olakšala, članovi su se sastajali jednom tjedno uživo i preko teamsa tokom praznika. Osim toga, napravljena je i whatsapp grupa u kojoj je u koju je voditeljica nakon sastanka slala zapisnik o sastanku i koja je služila kao glavni medij komunikacije među članovima.
        
        Sudjelovanje u ovakvom projektu predstavlja vrijedno iskustvo za sve članove tima. Kroz zajednički rad u iznimno dobro funkcionalnom timu stekli smo iskustvo koje nam u budućnosti može služiti kao primjer tima u kakvom želimo biti. Zajedno smo osvijestili važnost dobre vremenske organiziranosti i koordinacije među članovima tima, ali i osjećaj odgovornosti prema drugim člnovima koji nemamo u samostalnom radu i učenju. Zadovoljni smo postignutim rezultatima ali svjesni smo da ima puno prostora za napredak naših vještima, kako tehničkih tako i organizacijskih,timskih i komunikacijskim.

        Logičan korak u razvoju aplikacije koji bi mogao uslijediti nakon projekt je izrada aplikacije za mobilne uređaje.
		\eject 