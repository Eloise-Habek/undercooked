\chapter{Zaključak i budući rad}
		
		\textbf{\textit{dio 2. revizije}}\\
		
        Na ovom je projektu uspješno izrađena web aplikaciju za razmjenu recepata i diskusiju između korisnika. Faze projekta odvile su se u dvije faze, prva u prvom ciklusu semestra, druga u drugom.
        
        U prvoj fazi projekta, okupljen je tim od 7 članova. Najprije je odabrana tehnologija koja će se koristiti tijekom projekta te je napravljena podjela projektnih zadataka po interesima i sposobnostima članova, gdje je ujedno bio cilj da svi dođu u dodir sa svim fazama razvoja i da količina posla bude ravnomjerno podijeljena. Rana podjela odgovornosti u timu omogućila je članovima da rano krenu učiti o tehnologiji te istraživati odabrane alate i programskih jezika koju će oni koristiti. Uslijedio je rad na dokumentaciji i pokretanje razvoja aplikacije.
        U drugoj fazi projekta uslijedila je realizacija rješenja i izradba preostale dokumentacije. Kako bi se rad članova što više uskladio i komunikacija olakšala, članovi su se sastajali jednom tjedno uživo i preko teamsa tijekom praznika. Osim toga, napravljena je i whatsapp grupa u kojoj je u koju je voditeljica nakon sastanka slala zapisnik o sastanku i koja je služila kao glavni medij komunikacije među članovima.
        
        Kroz zajednički rad na ovom projektu u složnom i dobro organiziranom timu, svi su članovi stekli iskustvo koje im u budućnosti može služiti kao primjer tima u kakvom žele biti. Zajedno smo osvijestili važnost dobre vremenske organiziranosti i koordinacije među članovima tima, ali i osjećaj odgovornosti prema drugim članovima koji nemamo u samostalnom radu i učenju. Zadovoljni smo postignutim rezultatima ali svjesni smo da ima puno prostora za napredak naših vještina, kako tehničkih tako i organizacijskih,timskih i komunikacijskim.

        Logičan korak u razvoju projekta koji bi dalje mogao slijediti je ostvarenje dodatnih funkcionalnosti na web aplikaciji, dodatan rad na design-u aplikacije ili izrada iste aplikacije ali za mobilne uređaje.
		\eject 