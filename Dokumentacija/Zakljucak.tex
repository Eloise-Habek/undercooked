\chapter{Zaključak i budući rad}
		
		%\textbf{\textit{dio 2. revizije}}\\
		
        Na ovom je projektu uspješno izrađena web stranica za dijeljenje recepata i diskusiju između korisnika. Projekt se odvio u dvije glavne faze, prva faza se odvijala u prvom ciklusu semestra koju obilježava upoznavanje s projektom, te druga faza koja se odvijala tokom drugog ciklusa gdje smo se usredotočili na implementaciju web stranice.
        
        U prvoj fazi projekta, okupljen je tim od 7 članova. Najprije su odabrane tehnologije koje će se koristiti tijekom projekta te je napravljena podjela projektnih zadataka po interesima i sposobnostima članova, gdje je ujedno bio cilj da svi dođu u dodir sa svim fazama razvoja i da količina posla bude ravnomjerno podijeljena. Rana podjela odgovornosti u timu omogućila je članovima da rano krenu učiti o tehnologijama u kojima će raditi te istraživati odabrane alate i programske jezike. Uslijedio je rad na dokumentaciji i pokretanje razvoja aplikacije. U prvoj fazi razvoja web stranice najteži dio je bio definirati obrasce uporabe te funkcionalnosti kako se ne bi trebali vraćati popravljati greške zbog manjka diskusije i razmišljanja na početku. Sprječili smo ovakve probleme uz puno otvorenih timskih diskusija gdje smo uskladili ideje i očekivanja koje smo imali o projektu.
        U drugoj fazi projekta uslijedila je realizacija rješenja i izrada preostale dokumentacije. Kako bi se rad članova što više uskladio i komunikacija olakšala, članovi su se sastajali jednom tjedno uživo i preko teamsa tijekom praznika. Osim toga, napravljena je i Whatsapp grupa u kojoj je voditeljica nakon sastanaka slala zapisnik o sastanku i koja je služila kao glavni medij komunikacije među članovima.
        
        Kroz zajednički rad na ovom projektu u složnom i dobro organiziranom timu, svi su članovi stekli iskustvo koje im u budućnosti može služiti kao primjer tima u kakvom žele biti. Zajedno smo osvijestili važnost dobre vremenske organiziranosti i koordinacije među članovima tima, ali i osjećaj odgovornosti prema drugim članovima koji nemamo u samostalnom radu i učenju. Zadovoljni smo postignutim rezultatima ali svjesni smo da ima puno prostora za napredak naših vještina, kako tehničkih tako i organizacijskih,timskih i komunikacijskih.

        Logičan korak u razvoju projekta koji bi dalje mogao slijediti je ostvarenje dodatnih funkcionalnosti na web stranici, dodatan rad na UI stranice ili izrada mobilne aplikacije.
        
        \eject 