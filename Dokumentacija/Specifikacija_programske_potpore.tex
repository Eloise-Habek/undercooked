\chapter{Specifikacija programske potpore}
		
	\section{Funkcionalni zahtjevi}
			
			\textbf{\textit{dio 1. revizije}}\\
			
			\textit{Navesti \textbf{dionike} koji imaju \textbf{interes u ovom sustavu} ili  \textbf{su nositelji odgovornosti}. To su prije svega korisnici, ali i administratori sustava, naručitelji, razvojni tim.}\\
				
			\textit{Navesti \textbf{aktore} koji izravno \textbf{koriste} ili \textbf{komuniciraju sa sustavom}. Oni mogu imati inicijatorsku ulogu, tj. započinju određene procese u sustavu ili samo sudioničku ulogu, tj. obavljaju određeni posao. Za svakog aktora navesti funkcionalne zahtjeve koji se na njega odnose.}\\
			
			
			\noindent \textbf{Dionici:}
			
			\begin{packed_enum}
				
				\item Naručitelj 
				\item Razvojni tim
				\item Korisnici (kuhari i amaterski kuhari)
				\item Administrator
				
			\end{packed_enum}
			
			\noindent \textbf{Aktori i njihovi funkcionalni zahtjevi:}
			
			
			\begin{packed_enum}
				\item  \underbar{Neregistrirani korisnik (inicijator) može:}
				
				\begin{packed_enum}
					
					\item otvoriti novi korisnički račun za koji je potrebno unjeti korisničko ime koje nije već zauzeto, adresu elektroničke pošte i lozinku računa
					\item pretražiti recept po značajkama ili imenu
					\item pregledati recept
					\item osvježiti broj porcija, time će se s obzirom da broj unesem mjerice sastojaka ažurirati
					\item otvoriti profile registriranih korisnika koji su postavili recept
					
				\end{packed_enum}

				\item  \underbar{Registrirani korisnik (inicijator) može:}
				
				\begin{packed_enum}
					
					\item ažurirati profil, podaci koje je moguće ažurirati su email, broj telefona i vremenski period u kojima je korisnik dostupan
					\item objaviti recept s obaveznim navođenjem naziva recepta, sastojaka, postupka pripreme i ukupnog trajanja pripreme, mogućnost navođenja dodatnih karakteristika poput oznaka tipa jela i dodavanja slika i videozapisa
					\item ažuriranje već objavljenjih recepata
					\item spremanje recepata kako bi im mogli lakše pristupiti kasnije
					\item zapratiti drugog autora (registriranog korisnika), korisnik će dobiti obavijest kada autor objavi novi recept
					\item pregled vlastitog profila
					\item brisanje recepata sa svojeg profila
					\item pisanje komentara na recepte drugih korisnika
					\item slanje poruka u privatnom chatu s drugim korisnikom
					\item pregled poruka privatnih chatova
					\item ispisati recept
					
				\end{packed_enum}

				\item  \underbar{Administrator (inicijator) može:}
				\begin{packed_enum}

					\item brisanje, uređivanje i kreiranje korisničkog profila
					\item brisanje, uređivanje i mijenjanje oznaka recepata
					\item vidjeti popis svih registriranih korisnika, njihovih podataka i recepata
					\item pregled komentara na receptima
					\item pristupiti statistikama (npr. najpopularniji recept po broju profila koji su ga spremili, najaktivniji profil s najviše recepata, najpopularniji profil s najčešće spremljenim receptima)
					\item brisanje komentara koji nisu u skladu s pravilima ophođenja na platformi

				\end{packed_enum}
			
				\item  \underbar{Baza podataka (sudionik) može:}
				
				\begin{packed_enum}
					
					\item pohranjuje sve podatke o korisnicima i njihovim ovlastima
					\item pohranjuje sve podatke o receptima
					\item pohranjuje informacije o odnosima među korisnika i recepata
					
				\end{packed_enum}
			\end{packed_enum}
			
			\eject 
			
			
				
			\subsection{Obrasci uporabe}
				
				\textbf{\textit{dio 1. revizije}}
				
				\subsubsection{Opis obrazaca uporabe}
					\textit{Funkcionalne zahtjeve razraditi u obliku obrazaca uporabe. Svaki obrazac je potrebno razraditi prema donjem predlošku. Ukoliko u nekom koraku može doći do odstupanja, potrebno je to odstupanje opisati i po mogućnosti ponuditi rješenje kojim bi se tijek obrasca vratio na osnovni tijek.}\\
					

					\noindent \underbar{\textbf{UC$<$broj obrasca$>$ -$<$ime obrasca$>$}}
					\begin{packed_item}
	
						\item \textbf{Glavni sudionik: }$<$sudionik$>$
						\item  \textbf{Cilj:} $<$cilj$>$
						\item  \textbf{Sudionici:} $<$sudionici$>$
						\item  \textbf{Preduvjet:} $<$preduvjet$>$
						\item  \textbf{Opis osnovnog tijeka:}
						
						\item[] \begin{packed_enum}
	
							\item $<$opis korak jedan$>$
							\item $<$opis korak dva$>$
							\item $<$opis korak tri$>$
							\item $<$opis korak četiri$>$
							\item $<$opis korak pet$>$
						\end{packed_enum}
						
						\item  \textbf{Opis mogućih odstupanja:}
						
						\item[] \begin{packed_item}
	
							\item[2.a] $<$opis mogućeg scenarija odstupanja u koraku 2$>$
							\item[] \begin{packed_enum}
								
								\item $<$opis rješenja mogućeg scenarija korak 1$>$
								\item $<$opis rješenja mogućeg scenarija korak 2$>$
								
							\end{packed_enum}
							\item[2.b] $<$opis mogućeg scenarija odstupanja u koraku 2$>$
							\item[3.a] $<$opis mogućeg scenarija odstupanja  u koraku 3$>$
							
						\end{packed_item}
					\end{packed_item}
				
					
				\subsubsection{Dijagrami obrazaca uporabe}
					
					\textit{Prikazati odnos aktora i obrazaca uporabe odgovarajućim UML dijagramom. Nije nužno nacrtati sve na jednom dijagramu. Modelirati po razinama apstrakcije i skupovima srodnih funkcionalnosti.}
				\eject		
				
			\subsection{Sekvencijski dijagrami}
				
				\textbf{\textit{dio 1. revizije}}\\
				
				\textit{Nacrtati sekvencijske dijagrame koji modeliraju najvažnije dijelove sustava (max. 4 dijagrama). Ukoliko postoji nedoumica oko odabira, razjasniti s asistentom. Uz svaki dijagram napisati detaljni opis dijagrama.}
				\eject
	
		\section{Ostali zahtjevi}
		
			\textbf{\textit{dio 1. revizije}}\\
		 
			 \textit{Nefunkcionalni zahtjevi i zahtjevi domene primjene dopunjuju funkcionalne zahtjeve. Oni opisuju \textbf{kako se sustav treba ponašati} i koja \textbf{ograničenja} treba poštivati (performanse, korisničko iskustvo, pouzdanost, standardi kvalitete, sigurnost...). Primjeri takvih zahtjeva u Vašem projektu mogu biti: podržani jezici korisničkog sučelja, vrijeme odziva, najveći mogući podržani broj korisnika, podržane web/mobilne platforme, razina zaštite (protokoli komunikacije, kriptiranje...)... Svaki takav zahtjev potrebno je navesti u jednoj ili dvije rečenice.}
			 
			 
			 
	