\chapter{Arhitektura i dizajn sustava}
		
		\textbf{\textit{dio 1. revizije}}\\

		\textit{ Potrebno je opisati stil arhitekture te identificirati: podsustave, preslikavanje na radnu platformu, spremišta podataka, mrežne protokole, globalni upravljački tok i sklopovsko-programske zahtjeve. Po točkama razraditi i popratiti odgovarajućim skicama:}
	\begin{itemize}
		\item 	\textit{izbor arhitekture temeljem principa oblikovanja pokazanih na predavanjima (objasniti zašto ste baš odabrali takvu arhitekturu)}
		\item 	\textit{organizaciju sustava s najviše razine apstrakcije (npr. klijent-poslužitelj, baza podataka, datotečni sustav, grafičko sučelje)}
		\item 	\textit{organizaciju aplikacije (npr. slojevi frontend i backend, MVC arhitektura) }		
	\end{itemize}
		
		\section{Organizacija sustava}
			\subsection{Uvod}
				Arhitekturu sustava čine tri glavna dijela:
				\begin{packed_item}
					\item Baza podataka
					\item Internetski poslužitelj
					\item Aplikacija
				\end{packed_item}
			
				Internetski poslužitelj ključan je dio u uspostavljanju komunikacije između korisnika i aplikacije. Korisnik aplikaciji pristupa pomoću internetskog preglednika na svom računalu, mobitelu ili nekom drugom uređaju. Preglednik komunicira s poslužiteljem preko HTTP protokola slanjem odgovarajućih zahtjeva. Poslužitelj je onaj koji pokreće aplikaciju te joj prosljeđuje korisnikove HTTP zahtjeve.
				\linebreak
				Aplikacija preuzima zahtjev te ga obrađuje sukladno njegovoj vrsti i parametrima. Obrada zahtjeva uključuje i pristupanje bazi podataka kako bi se dohvatili podatci potrebni za rad. Po završetku obrade zahtjeva, aplikacija preko poslužitelja korisniku vraća odgovor.
				\linebreak
				Baza podataka može se nalaziti na istom računalu kao i poslužitelj i aplikacija ili različitom, no komunikacija se uvijeko odvija na isti način, preko preddefiniranih vrata na transportnom sloju i korištenjem odgovarajućeg protokola.
				\begin{figure}[H]
					\includegraphics[scale=1]{slike/skica_arhitekture.png}
					\centering
					\caption{Arhitektura sustava}
					\label{fig:arhitektura_sustava}
				\end{figure}
			\subsection{Sklopovski zahtjevi}
				Ukoliko se poslužitelj i baza podataka nalaze na različitim računalima, za optimalan rad ona bi trebala imati odgovarajuće karakteristike. Računalo na kojem će raditi poslužitelj treba imati dovoljnu veliku procesorsku moć kako bi moglo što brže odgovarati na zahtjeve i kako bi više korisnika moglo koristiti aplikaciju bez značajnog usporavanja sustava. Računalo na kojem će se nalaziti baza podataka treba imati dovoljno velike i brze diskove za pohranu podataka, idealno uz neku vrstu zaštite od gubitka podataka u slučaju kvara (na primjer, korištenje RAID sustava ili automatskog redovitog stvaranja sigurnosne kopije). Ako se pak poslužitelj i baza podataka pokreću na istom računalu, ono treba kombinirati prethodno navedene karakteristike.
			\subsection{Organizacija aplikacije}
				Za izradu aplikacije odabrani su programski jezik Java uz razvojni okvir Spring te Javascript i razvojni okvir React. Dva glavna sloja aplikacije su frontend, koji komunicira s korisnikom (Javascript + React) i backend, koji obrađuje HTTP zahtjeve i komunicira s bazom podataka (Java + Spring).
				%U primjeru dokumentacije oni ovdje jos pisu da se arhitektura temelji na MVCu i to dodatno opisuju
				%Ovaj dio bi onda trebali dovrsit kada vidimo sto cemo s time
				
				
		

		\section{Baza podataka}
			
    	       Naš će sustav koristiti relacijsku bazu podataka čija je osnovna jedinica baze relacija, odnosno tablica, koja je definirana svojim imenom i skupom atributa. Baza podataka ima zadatak brzo i jednostavno pohranjivati, mijenjati i dohvaćati podatke za daljnju obradu. Baza podataka stranice sastoji se od sljedećih entiteta:
                \begin{packed_item}
					\item Korisnik
					\item Recept
					\item Oznaka
                        \item Komentar
					\item Ocjena
					\item Poruka
                        \item Sastojak
				\end{packed_item}
		
			\subsection{Opis tablica}
			

				\textbf{Korisnik} Ovaj entitet sadrzava sve važne informacije o korisniku aplikacije. Sadrzi atribute: Korisnicko ime, lozinku, ime, prezime, broj mobitela korisnika, razinu ovlasti korisnika i e-mail korisnika. Ovaj entitet u vezi je \textit{One-to-Many} sa entitetima Recept, Komentar i Ocjena te je sa entitetom poruka u ulozi primatelj i pošiljatelj isto u \textit{One-to-Many} vezi.
				
				
				\begin{longtblr}[
					label=none,
					entry=none
					]{
						width = \textwidth,
						colspec={|X[8,l]|X[8, l]|X[18, l]|}, 
						rowhead = 1,
					} %definicija širine tablice, širine stupaca, poravnanje i broja redaka naslova tablice
					\hline \SetCell[c=3]{c}{\textbf{Korisnik}}	 \\ \hline[3pt]
					\SetCell{LightGreen} ID Korisnika	& INT &  jedinstveni identifikator korisnika\\ \hline 
					Lozinka & VARCHAR	&  lozinka korisnika	\\ \hline 
                        Korisničko ime & VARCHAR	&  korisničko ime	\\ \hline 
                        Ime & VARCHAR	& ime korisnika\\ \hline 
					Prezime & VARCHAR &  prezime korisnika \\ \hline 
                        Broj mobitela & VARCHAR	& telefonski broj korisnika\\ \hline 
                        Razina Ovlasti & VARCHAR &  razina ovlasti korisnika \\ \hline 
					Email & VARCHAR & e-mail adresa korisnika 	\\ \hline 
				\end{longtblr}
				
				\textbf{Recept} Ovaj entitet sadrzava sve važne informacije o receptu. Sadrzi atribute: ID recepta, naziv recepta, vrijeme pripreme, postupak pripreme, opis recepta, sliku recepta, datum recepta  i vrijeme objave recepta i prosječnu ocjenu recepta. Recept je u vezi \textit{Many-to-One} sa korisnikom koji ga je objavio, u vezi \textit{One-to-Many} sa ocjenom, sastojkom u receptu i komentarom te u \textit{Many-to-Many} sa oznakom recepta.
				
				\begin{longtblr}[
					label=none,
					entry=none
					]{
						width = \textwidth,
						colspec={|X[8,l]|X[8, l]|X[18, l]|}, 
						rowhead = 1,
					} %definicija širine tablice, širine stupaca, poravnanje i broja redaka naslova tablice
					\hline \SetCell[c=3]{c}{\textbf{Recept}}	 \\ \hline[3pt]
					\SetCell{LightGreen}ID Recepta & INT	&  	jedinstveni identifikator recepta  	\\ \hline
					Naziv recepta	& VARCHAR &   naziv recepta	\\ \hline 
					Vrijeme pripreme & INTERVAL & vrijeme pripreme  \\ \hline 
					Postupak pripreme & VARCHAR	& postupak pripreme\\ \hline 
					Opis recepta & VARCHAR & opis recepta \\ \hline 
					Slika recepta & LONGBOB	&  slika recepta	\\ \hline 
                        Datum i vrijeme recepta	& DATETIME & datum recepta  i vrijeme objave recepta 	\\ \hline 
                        Prosječna ocjena recepta	& INT &   prosječna ocjena recepta	\\ 
                        \hline
                        \SetCell{LightBlue} ID Korisnika	& INT &  ID korisnika koji je objavio recept\\ \hline 
                        \SetCell{LightBlue} ID Oznake	& INT &  ID oznake recepta\\ \hline 
				\end{longtblr}

                \textbf{Oznaka} Ovaj entitet sadrzava sve važne informacije o oznaci recepta. Sadrzi atribute: ID oznake i naziv oznake. Oznaka je u \textit{Many-to-Many} vezi s receptom.
    
                \begin{longtblr}[
					label=none,
					entry=none
					]{
						width = \textwidth,
						colspec={|X[8,l]|X[8, l]|X[18, l]|}, 
						rowhead = 1,
					} %definicija širine tablice, širine stupaca, poravnanje i broja redaka naslova tablice
					\hline \SetCell[c=3]{c}{\textbf{Oznaka}}	 \\ \hline[3pt]
					\SetCell{LightGreen}ID oznake & INT	&  	jedinstveni identifikator oznake  	\\ \hline
					Naziv oznake & VARCHAR & naziv oznake  	\\ \hline 
				\end{longtblr}

                \textbf{Komentar} Ovaj entitet sadrzava sve važne informacije o komentaru recepta. Sadrzi atribute: ID komentatora, ID recepta, tekst komentara i datum i vrijeme komentara. Komentar je u \textit{Many-to-One} vezi s korisnikom koji ga objavi i receptom. 

                \begin{longtblr}[
					label=none,
					entry=none
					]{
						width = \textwidth,
						colspec={|X[8,l]|X[8, l]|X[18, l]|}, 
						rowhead = 1,
					} %definicija širine tablice, širine stupaca, poravnanje i broja redaka naslova tablice
					\hline \SetCell[c=3]{c}{\textbf{Komentar}}	 \\ \hline[3pt]
					\SetCell{LightBlue}Tekst komentara	& VARCHAR &  tekst komentara 	\\ \hline 
                        \SetCell{LightBlue}ID komentatora	& INT &  ID korisnika komentatora	\\ \hline
                        \SetCell{LightBlue}ID recepta	& INT &  ID recepta na kojem je komentar postavljen	\\ \hline
                        \SetCell{LightBlue}Datum komentara	& DATETIME &  datum i vrijeme komentara	\\ \hline 
				\end{longtblr}

                \textbf{Ocjena} Ovaj entitet sadrzava sve važne informacije o pojedinoj ocjeni recepta. Sadrzi atribute: ID ocjenitelja, ID recepta i datum i vrijeme ocjene. Entitet je u vezi \textit{Many-to-One} s ocjeniteljem i \textit{Many-to-One} s receptom.

                \begin{longtblr}[
					label=none,
					entry=none
					]{
						width = \textwidth,
						colspec={|X[8,l]|X[8, l]|X[18, l]|}, 
						rowhead = 1,
					} %definicija širine tablice, širine stupaca, poravnanje i broja redaka naslova tablice
					\hline \SetCell[c=3]{c}{\textbf{Ocjena}}	 \\ \hline[3pt]
					Ocjena	& VARCHAR & ocjena	\\ \hline
                        \SetCell{LightBlue}ID ocjenitelja	& INT &  ID korisnika koji je ostavio poruku	\\ \hline
                        \SetCell{LightBlue}ID recepta	& INT &  ID recepta koji je ocjenjen	\\ \hline
                        Datum i vrijeme ocjene	& DATETIME &  datum i vrijeme ocjene	\\ \hline 
				\end{longtblr}

                \textbf{Poruka} Ovaj entitet sadrzava sve važne informacije o poruci između dva korisnika. Sadrzi atribute: ID pošiljatelja, ID primatelja, tekst poruke i datum i vrijeme poruke. Poruka je u vezi \textit{Many-to-One} s primateljem i pošiljateljem.

                \begin{longtblr}[
					label=none,
					entry=none
					]{
						width = \textwidth,
						colspec={|X[8,l]|X[8, l]|X[18, l]|}, 
						rowhead = 1,
					} %definicija širine tablice, širine stupaca, poravnanje i broja redaka naslova tablice
					\hline \SetCell[c=3]{c}{\textbf{Poruka}}	 \\ \hline[3pt]
					\SetCell{LightBlue}Tekst poruka & VARCHAR & tekst poruka  	\\ \hline 
                        \SetCell{LightBlue}ID pošiljatelja	& INT &  ID korisnika koji šalje poruku	\\ \hline 
                        \SetCell{LightBlue}ID primatelja	& INT & ID korisnika koji prima poruku	\\ \hline 
                        \SetCell{LightBlue}Datum i vrijeme poruke & TIMESTAMP &  datum i vrijeme poruke	\\ \hline 
				\end{longtblr}

                \textbf{Sastojak u receptu} Ovaj entitet sadrzava sve važne informacije o sastojku navedenom u nekom receptu. Sadrzi atribute: ID recepta, naziv sastojka, količinu. Entitet je u vezi \textit{Many-to-One} sa receptom.
                
                \begin{longtblr}[
					label=none,
					entry=none
					]{
						width = \textwidth,
						colspec={|X[8,l]|X[8, l]|X[18, l]|}, 
						rowhead = 1,
					} %definicija širine tablice, širine stupaca, poravnanje i broja redaka naslova tablice
                    %Zašto imamo IDSastojak? Jer može biti više sastojaka s istim imenom, istom količinom u istom receptu (npr 100 ml mlijeka u kremi i 100 ml mlijeka u tjestu)
					\hline \SetCell[c=3]{c}{\textbf{Sastojak}}	 \\ \hline[3pt]
                        \SetCell{LightGreen}ID Sastojka	& INT &  ID sastojka \\ \hline
                        Naziv sastojka	& VARCHAR &  naziv sastojka	\\ \hline
                        \SetCell{LightBlue}ID recepta	& INT &   ID recepta u kojem je sastojak	\\ \hline
                        Količina	& VARCHAR &   količina	\\ \hline
					
				\end{longtblr}
				
			
			\subsection{Dijagram baze podataka}
			\begin{figure}[H]
				\includegraphics[scale=0.15]{slike/ER diagram baze podataka.png} %veličina slike u odnosu na originalnu datoteku i pozicija slike
				\centering
				\caption{ER dijagram baze podataka}
				\label{fig:ER_diagram}
			\end{figure}
		
			\eject
			
			
		\section{Dijagram razreda}
		
			\textit{Potrebno je priložiti dijagram razreda s pripadajućim opisom. Zbog preglednosti je moguće dijagram razlomiti na više njih, ali moraju biti grupirani prema sličnim razinama apstrakcije i srodnim funkcionalnostima.}\\
			
			\textbf{\textit{dio 1. revizije}}\\
			
			\textit{Prilikom prve predaje projekta, potrebno je priložiti potpuno razrađen dijagram razreda vezan uz \textbf{generičku funkcionalnost} sustava. Ostale funkcionalnosti trebaju biti idejno razrađene u dijagramu sa sljedećim komponentama: nazivi razreda, nazivi metoda i vrste pristupa metodama (npr. javni, zaštićeni), nazivi atributa razreda, veze i odnosi između razreda.}\\
			
			\textbf{\textit{dio 2. revizije}}\\			
			
			\textit{Prilikom druge predaje projekta dijagram razreda i opisi moraju odgovarati stvarnom stanju implementacije}
			
			
			
			\eject
		
		\section{Dijagram stanja}
			
			
			\textbf{\textit{dio 2. revizije}}\\
			
			\textit{Potrebno je priložiti dijagram stanja i opisati ga. Dovoljan je jedan dijagram stanja koji prikazuje \textbf{značajan dio funkcionalnosti} sustava. Na primjer, stanja korisničkog sučelja i tijek korištenja neke ključne funkcionalnosti jesu značajan dio sustava, a registracija i prijava nisu. }
			
			
			\eject 
		
		\section{Dijagram aktivnosti}
			
			\textbf{\textit{dio 2. revizije}}\\
			
			 \textit{Potrebno je priložiti dijagram aktivnosti s pripadajućim opisom. Dijagram aktivnosti treba prikazivati značajan dio sustava.}
			
			\eject
		\section{Dijagram komponenti}
		
			\textbf{\textit{dio 2. revizije}}\\
		
			 \textit{Potrebno je priložiti dijagram komponenti s pripadajućim opisom. Dijagram komponenti treba prikazivati strukturu cijele aplikacije.}